\hspace{0.45cm} \textbf{Спектр сингулярного разложения} (сингулярный спектр) - это набор всех сингулярных значений: \( \operatorname{diag}(\Sigma) = \{ \sigma_1, \sigma_2, \dots, \sigma_{\min(m, n)} \} \).

\textbf{Нахождение полного SVD} - нахождение по первоначальной матрице правого и левого сингулярных векторов и диагональной матрицы сингулярных значений. 

\textbf{Итеративный алгоритм} - это такой алгоритм, условие схождения которого таково, что нельзя однозначно сказать, сколько итераций будет произведено в конкретном случае. Далее в тексте при использовании слова. 

