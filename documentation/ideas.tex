\documentclass[12pt, a4paper]{report}

\usepackage{listings}
\usepackage{xcolor}
\usepackage{ragged2e}
\usepackage{xurl}

\usepackage[a4paper, top=2cm, bottom=2cm, left=3cm, right=3cm]{geometry}

\usepackage[T2A]{fontenc}
\usepackage[utf8]{inputenc}
\usepackage[russian]{babel}

\usepackage{hyperref}
\usepackage{amsmath}
\usepackage{cleveref}
\usepackage{csquotes}
\usepackage{amsmath}
\usepackage{amsfonts}
\usepackage{amsthm}

\usepackage[nonumberlist, acronym]{glossaries}

\usepackage{tcolorbox}
\tcbuselibrary{listings, breakable}

\newtcblisting{mytcblisting}[2][]{%
  listing only,
  breakable,
  colframe=black,
  colback=white,
  boxsep=5pt,
  title={#2},
  listing options={
    basicstyle=\ttfamily\small,
    breaklines=true,
    breakatwhitespace=true,
    postbreak=\mbox{\textcolor{red}{$\hookrightarrow$}\space},
    columns=fullflexible,
    keepspaces=true,
    language=C,
    numbers=left,
    numbersep=10pt,
    stepnumber=1,
    firstnumber=1,
    numberstyle=\tiny\color{gray}
  },
  #1
}

\bibliographystyle{plain}

\hypersetup{
    colorlinks=true,
    linkcolor=black,
    citecolor=black,
    filecolor=black,
    urlcolor=black
}

\newtheoremstyle{note}
    {3pt}
    {3pt}
    {}
    {1.25cm}
    {\bfseries}
    {.}
    { }
    {}

% Несколько полезных блоков для выделения важных частей текста. Пример:
% \begin{theorem}
%   Если $a$ и $b$ — положительные числа, то $a + b \geq 2\sqrt{ab}$.
% \end{theorem}
\theoremstyle{note}
\newtheorem*{note}{Замечание}
\newtheorem{theorem}{Теорема}
\newtheorem{lemma}{Лемма}
\newtheorem{corollary}{Следствие}
\newcommand{\R}[2]{\mathbb{R}^{#1 \times #2}}
\newcommand{\Rn}{\mathbb{R}^n}

\begin{document}

\tableofcontents

\chapter{Уточнение поворотов Гивенса в implicit zero-shift QR}

\section{Схождение алгоритма}
В данном разделе рассмотрим критерии сходимости для нашего алгоритма приведения бидиагольной матрицы $B$ к диагональной $\Sigma$. Пусть $s_1, s_2,...,s_n$ \--- диагональные элементы, а $e_1, e_2, ..., e_{n-1}$ \--- элементы на побочной диагонали нашей матрицы $B$. Также есть некоторый критерий допуска $tol$, который зависит от желаемой относительной точности сингулярных значений. Он должен быть меньше 1, но больше машинной точности \epsilon.

Наш критерий сходимости должен гарантировать, что обнуление $e_i$ сильно не повлияет на сингулярные значения. Пусть $\sigma$ обозначет нижнюю границу для наименьшего сингулярного значения, тогда самый простой допустимый вариант будет установка $e_i$ равными 0, если значение меньше, чем $tol\cdot\sigma$. Однако, при таком методе, наши числа на побочной диагонали будут зануляться очень долго. Гораздо лучшие критерии можно получить такими способами:

Пусть $\lambda_j$ и $\mu_j$ вычисляются с помощью данных реккурентных соотношений:

\begin{minipage}{0.48\textwidth}
\begin{align*}
\mu_1& = |s_1| \\
\text{for }& j = 1 \text{ to } n-1 \text{ do} \\
&\mu_{j+1} = |s_{j+1}| \cdot \left( \frac{\mu_j}{\mu_j + |e_j|} \right)
\end{align*}
\end{minipage}
\hfill
\begin{minipage}{0.48\textwidth}
\begin{align*}
\lambda_n& = |s_n| \\
\text{for }& j = n-1 \text{ to } 1 \text{ step } -1 \text{ do} \\
&\lambda_j = |s_j| \cdot \left( \frac{\lambda_{j+1}}{\lambda_{j+1} + |e_j|} \right)
\end{align*}
\end{minipage}
\vspace{1em}

\noindent\textit{Критерий сходимости 1a}. Если $|\frac{e_j}{\mu_j}|\leq tol$, то обнуляем $e_j$.\vspace{1em}
\\\textit{Критерий сходимости 1b}. Если $|\frac{e_j}{\lambda_{j+1}}|\leq tol$, то обнуляем $e_j$.\vspace{1em}
\\\textit{Критерий сходимости 2a}. Если сингулярные вектора не требуеются и\linebreak $e^2_{n-1}\leq0.5\cdot tol\cdot [(\frac{\min\limits_{j<n}\mu_j}{\sqrt{n-1}})^2-|s_n|^2]$, то обнуляем $e_{n-1}$.\vspace{1em}
\\\textit{Критерий сходимости 2b}. Если сингулярные вектора не требуеются и\linebreak $e^2_1\leq0.5\cdot tol\cdot [(\frac{\min\limits_{j>1}\lambda_j}{\sqrt{n-1}})^2-|s_1|^2]$, то обнуляем $e_1$.\vspace{1em}

Эти критерии требуют больше вычислительных затрат, чем критерии, реализованные в библиотеке LINPACK, зато помогают избежать ситуаций, когда обнуление значения приводит к недопустимой относительной ошибке.


% Способы определения схождения алгоритма на этапе "подкрутки алгоритмов". lawn03, страница 17
\chapter{"обратный" поворот Якоби}

\section{Принцип сведения $A$ к \Sigma}
Ну в общем бам бам и все готовт
% Описать работу первого шага алгоритма
% раздел 12 https://pure.manchester.ac.uk/ws/files/82231456/17m1117732.pdf 
% golub van loan 8.6

\chapter{"Наивный" метод}

\chapter{Скрученные преобразования}

\end{document}