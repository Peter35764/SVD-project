\hspace{0.45cm} \textbf{Сингулярное разложение (SVD)} матрицы \( A \) размера \( m \times n \) записывается в виде:
\[
A = U \Sigma V^T \quad (\text{или} \quad A = U \Sigma V^H \ \text{в комплексном случае}),
\]
где:
\begin{itemize}
    \item \( U \) и \( V \) — ортогональные (или унитарные в комплексном случае) матрицы;
    \item Столбцы \( U \) и \( V \) ортонормированы относительно себя, то есть:
    \[
    U^T U = I \text{ и } V^T V = I \quad \text{(вещественный случай)},
    \]
    \[
    U^H U = I \text{ и } V^H V = I \quad \text{(комплексный случай)};
    \]
    \item \( U \) и \( V \) \textbf{биортогональны относительно друг друга через сингулярные значения}, то есть:
    \[
    U^T A V = \Sigma \quad \text{(вещественный случай)},
    \]
    \[
    U^H A V = \Sigma \quad \text{(комплексный случай)}.
    \]
    где \( \Sigma \) — диагональная матрица сингулярных значений;
    \item \( \Sigma \) — матрица размера \( m \times n \) с вещественными значениями \( \sigma_i \) на главной диагонали (\( \sigma_1 \) на позиции \( \Sigma_{11} \)), причем:
    \[
    \sigma_1 \geq \sigma_2 \geq \cdots \geq \sigma_{\min(m, n)} \geq 0.
    \]
\end{itemize}

Эти числа называются \textit{сингулярными значениями} матрицы A, первые 

\(\min(m,n)\) столбцы матриц U и V это \textit{левые} и \textit{правые сингулярные векторы} матрицы A.

\textbf{Спектр сингулярного разложения} (сингулярный спектр) - это набор всех сингулярных значений: \( \operatorname{diag}(\Sigma) = \{ \sigma_1, \sigma_2, \dots, \sigma_{\min(m, n)} \} \).

\textbf{Нахождение полного SVD} - нахождение по первоначальной матрице правого и левого сингулярных векторов и диагональной матрицы сингулярных значений. 

\textbf{Итеративный алгоритм} - это такой алгоритм, условие схождения которого таково, что нельзя однозначно сказать, сколько итераций будет произведено в конкретном случае. Далее в тексте при использовании слова. 

\textbf{Эрмитово сопряженная матрица} - Если исходная матрица $A$  имеет размер $m \times n$, то эрмитово сопряжённая к $A$ матрица $A^*$ будет иметь размер $n \times m$, а её $(i, j)$-й элемент будет равен:
\begin{center}
$(A^*)_{ij} = \overline{A_{ji}}$
\end{center}

\textbf{Нормальные матрицы} - матрица \(A\) нормальна, если \(A^*A=AA^*\).

\textbf{Унитарные матрицы} - это обобщение ортогональных матриц на комплексных случай. Если ортогональные матрицы таковы, что \(A^TA=AA^T=I\), то в случае унитарных матриц:
\[A^*A=AA^*=I,\]
где \(*\) - эрмитово сопряжение. Любая унитарная матрица является нормальной по определению.

\textbf{Спектральное разложение}. Пусть нам дана квадратная матрица \(A\). Тогда она может быть представлена в виде своего спектрального разложения:
\[ A=V\Lambda V^{-1},\]
 где \(V\) - это собственные векторы, \(\Lambda\) - собственные значения, спектр.

Квадратная матрица является \textbf{нормальной}, если её разложение представимо в виде:
\[A=Q \Lambda Q^*,\]
где \(Q\) - унитарная матрица. 

\textbf{Поворот Якоби или Гивенса} - это матрица вида
\[ \label{eq:2:1}
    J(i,j,\theta) = 
    \begin{pmatrix}E&&&&\\
        &c&&s\\
        &&E&&\\
        &-s&&c\\
        &&&&E
    \end{pmatrix},\
    c = \cos(\theta), s =\sin(\theta), 
\]
где значения отличаются от \(E\ - \) единичной матрицы в строках \(i, j\).

% TODO забагованно, не трогать, я исправлю
% 
% \textbf{Циклический обход матрицы} - это прохождение элементов матрицы размера $n \times n$ в определенном порядке \(\frac{n(n-1)}{2}\) комбинаций.
% В случае циклического обхода строк:
% \begin{center}
% ($i_0$, $j_0$) &= (1, 2), \tag{III_r} \\
% ($i_{k+1}$,$j_{k+1}$) &= 
% \begin{cases} 
% (i_k, j_k + 1), & \text{если } i_k < n-1, j_k < n, \\ 
% (i_k + 1, i_k + 2), & \text{если } i_k < n-1, j_k = n, \\ 
% (1, 2), & \text{если } i_k = n-1, j_k = n; 
% \end{cases}
% \end{center}
% В случае циклического обхода стобцов:
% \begin{center}
% ($i_0$, $j_0$) &= (1, 2), \\
% ($i_{k+1}$, $j_{k+1}$) &= 
% \begin{cases} 
% (i_k + 1, j_k), & \text{если } i_k < j_k - 1, j_k \leq n, \\
% (1, j_k + 1), & \text{если } i_k = j_k - 1, j_k < n, \\
% (1, 2), & \text{если  } i_k = n - 1, j_k = n.
% \end{cases}
% \end{center}
