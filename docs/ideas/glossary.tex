\hspace{0.45cm} \textbf{Спектр сингулярного разложения} (сингулярный спектр) - это набор всех сингулярных значений: \( \operatorname{diag}(\Sigma) = \{ \sigma_1, \sigma_2, \dots, \sigma_{\min(m, n)} \} \).

\textbf{Нахождение полного SVD} - нахождение по первоначальной матрице правого и левого сингулярных векторов и диагональной матрицы сингулярных значений. 

\textbf{Итеративный алгоритм} - это такой алгоритм, условие схождения которого таково, что нельзя однозначно сказать, сколько итераций будет произведено в конкретном случае. Далее в тексте при использовании слова. 

\textbf{Эрмитово сопряжение} - это обобщение операции транспонирования на комплексный случай. Мы транспонируем матрицу и накладываем комплексное сопряжение на все его элементы, то есть \(A^*=\overline{A^T}\).

\textbf{Нормальные матрицы} - матрица \(A\) нормальна, если \(A^*A=AA^*\).

\textbf{Унитарные матрицы} - это обобщение ортогональных матриц на комплексных случай. Если ортогональные матрицы таковы, что \(A^TA=AA^T=I\), то в случае унитарных матриц:
\[A^*A=AA^*=I,\]
где \(*\) - эрмитово сопряжение. Любая унитарная матрица является нормальной по определению.

\textbf{Спектральное разложение}. Пусть нам дана квадратная матрица \(A\). Тогда она может быть представлена в виде своего спектрального разложения:
\[ A=V\Lambda V^{-1},\]
 где \(V\) - это собственные векторы, \(\Lambda\) - собственные значения, спектр.

Квадратная матрица является нормальной, если её разложение представимо в виде:
\[A=Q \Lambda Q^*,\]
где \(Q\) - унитарная матрица. 
