\hspace{0.45cm} \textbf{Сингулярное разложение (SVD)} матрицы \( A \) размера \( m \times n \) записывается в виде:
\[
A = U \Sigma V^T \quad (\text{или} \quad A = U \Sigma V^H \ \text{в комплексном случае}),
\]
где:
\begin{itemize}
    \item \( U \) и \( V \) — ортогональные (или унитарные в комплексном случае) матрицы;
    \item Столбцы \( U \) и \( V \) ортонормированы относительно себя, то есть:
    \[
    U^T U = I \text{ и } V^T V = I \quad \text{(вещественный случай)},
    \]
    \[
    U^H U = I \text{ и } V^H V = I \quad \text{(комплексный случай)};
    \]
    \item \( U \) и \( V \) \textbf{биортогональны относительно друг друга через сингулярные значения}, то есть:
    \[
    U^T A V = \Sigma \quad \text{(вещественный случай)},
    \]
    \[
    U^H A V = \Sigma \quad \text{(комплексный случай)}.
    \]
    где \( \Sigma \) — диагональная матрица сингулярных значений;
    \item \( \Sigma \) — матрица размера \( m \times n \) с вещественными значениями \( \sigma_i \) на главной диагонали (\( \sigma_1 \) на позиции \( \Sigma_{11} \)), причем:
    \[
    \sigma_1 \geq \sigma_2 \geq \cdots \geq \sigma_{\min(m, n)} \geq 0.
    \]
\end{itemize}

Эти числа называются \textit{сингулярными значениями} матрицы A, первые 

\(\min(m,n)\) столбцы матриц U и V это \textit{левые} и \textit{правые сингулярные векторы} матрицы A.

\textbf{Спектр сингулярного разложения} (сингулярный спектр) - это набор всех сингулярных значений: \( \operatorname{diag}(\Sigma) = \{ \sigma_1, \sigma_2, \dots, \sigma_{\min(m, n)} \} \).

\textbf{Нахождение полного SVD} - нахождение по первоначальной матрице правого и левого сингулярных векторов и диагональной матрицы сингулярных значений. 

\textbf{Итеративный алгоритм} - это такой алгоритм, условие схождения которого таково, что нельзя однозначно сказать, сколько итераций будет произведено в конкретном случае. Далее в тексте при использовании слова. 

\textbf{Эрмитово сопряжение} - это обобщение операции транспонирования на комплексный случай. Мы транспонируем матрицу и накладываем комплексное сопряжение на все его элементы, то есть \(A^*=\overline{A^T}\).

\textbf{Нормальные матрицы} - матрица \(A\) нормальна, если \(A^*A=AA^*\).

\textbf{Унитарные матрицы} - это обобщение ортогональных матриц на комплексных случай. Если ортогональные матрицы таковы, что \(A^TA=AA^T=I\), то в случае унитарных матриц:
\[A^*A=AA^*=I,\]
где \(*\) - эрмитово сопряжение. Любая унитарная матрица является нормальной по определению.

\textbf{Спектральное разложение}. Пусть нам дана квадратная матрица \(A\). Тогда она может быть представлена в виде своего спектрального разложения:
\[ A=V\Lambda V^{-1},\]
 где \(V\) - это собственные векторы, \(\Lambda\) - собственные значения, спектр.

Квадратная матрица является нормальной, если её разложение представимо в виде:
\[A=Q \Lambda Q^*,\]
где \(Q\) - унитарная матрица. 

\textbf{Поворот Якоби или Гивенса} - это матрица вида
\[ \label{eq:2:1}
    J(i,j,\theta) = 
    \begin{pmatrix}E&&&&\\
        &c&&s\\
        &&E&&\\
        &-s&&c\\
        &&&&E
    \end{pmatrix},\
    c = \cos(\theta), s =\sin(\theta), 
\]
где значения отличаются от \(E\ - \) единичной матрицы в строках \(i, j\).
