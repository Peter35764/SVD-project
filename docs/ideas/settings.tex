\documentclass[12pt, a4paper]{report}

\usepackage{listings}
\usepackage{xcolor}
\usepackage{ragged2e}
\usepackage{xurl}

\usepackage[a4paper, top=2cm, bottom=2cm, left=3cm, right=3cm]{geometry}

\usepackage[T2A]{fontenc}
\usepackage[utf8]{inputenc}
\usepackage[russian]{babel}

\usepackage[backend=biber, style=numeric]{biblatex}
\addbibresource{references.bib}
% plain — default.
% alphabetic — буквенные метки [Gol96].
% ieee — стиль IEEE (требует пакет biblatex-ieee).

% \usepackage{glossaries}
% \newglossaryentry{svd}{name=SVD, description={Сингулярное разложение матрицы}}
% \printglossaries

\usepackage{hyperref}
\usepackage{amsmath}
\usepackage{cleveref}
\usepackage{csquotes}
\usepackage{amsmath}
\usepackage{amsfonts}
\usepackage{amsthm}

\usepackage[nonumberlist, acronym]{glossaries}

\usepackage{tcolorbox}

\setcounter{tocdepth}{4}

\tcbuselibrary{listings, breakable}

\newtcblisting{mytcblisting}[2][]{%
  listing only,
  breakable,
  colframe=black,
  colback=white,
  boxsep=5pt,
  title={#2},
  listing options={
    basicstyle=\ttfamily\small,
    breaklines=true,
    breakatwhitespace=true,
    postbreak=\mbox{\textcolor{red}{$\hookrightarrow$}\space},
    columns=fullflexible,
    keepspaces=true,
    language=C,
    numbers=left,
    numbersep=10pt,
    stepnumber=1,
    firstnumber=1,
    numberstyle=\tiny\color{gray}
  },
  #1
}

\hypersetup{
    colorlinks=true,
    linkcolor=black,
    citecolor=black,
    filecolor=black,
    urlcolor=black
}

\newtheoremstyle{note}
    {3pt}
    {3pt}
    {}
    {1.25cm}
    {\bfseries}
    {.}
    { }
    {}

% Несколько полезных блоков для выделения важных частей текста. Пример:
% \begin{theorem}
%   Если $a$ и $b$ — положительные числа, то $a + b \geq 2\sqrt{ab}$.
% \end{theorem}
\theoremstyle{note}
\newtheorem{example}{Пример}
\newtheorem{claim}{Утверждение} 
\newtheorem*{note}{Замечание}
\newtheorem{theorem}{Теорема}
\newtheorem{lemma}{Лемма}
\newtheorem{corollary}{Следствие}
\newcommand{\R}[2]{\mathbb{R}^{#1 \times #2}}
\newcommand{\Rn}{\mathbb{R}^n}

% Дополнительный уровень иерархии. Не попадает в оглавление
\newcommand{\subsubsubsection}[1]{%
  \paragraph*{\bfseries #1}             % Жирный заголовок
  \addcontentsline{toc}{paragraph}{#1}  % Добавляем в оглавление
  \mbox{}%                              % Пустая строка для переноса
  \par%                                 % Принудительный перенос строки
  \noindent%                            % Убирает отступ у следующего абзаца
}
