% Кодировка и язык
\usepackage[T2A]{fontenc} % поддержка кириллицы
\usepackage[utf8]{inputenc} % кодировка исходного текста
\usepackage[english,russian]{babel} % переключение языков

% Геометрия страницы и графика
\usepackage[left=3cm, right=1cm, top=2cm, bottom=2cm]{geometry} % поля страницы
\usepackage{graphicx} % подключение графики
\usepackage{pdfpages} % вставка pdf-страниц

% Таблицы
\usepackage{array} % расширенные возможности для работы с таблицами
\usepackage{tabularx} % автоматический подбор ширины столбцов
\usepackage{dcolumn} % выравнивание чисел по разделителю

% Математика
\usepackage{bm} % полужирное начертание для математических символов
\usepackage{amsmath} % дополнительные математические возможности
\usepackage{amssymb} % дополнительные математические символы
\usepackage{amsthm}       % основной пакет для работы с теоремами
\usepackage{amsfonts}     % математические шрифты (для \mathbb)

% Библиография и ссылки
\usepackage{cite} % поддержка цитирования
\usepackage{hyperref} % создание гиперссылок

% Прочее
\usepackage{color} % работа с цветом
\usepackage{epstopdf} % конвертация eps в pdf
\usepackage{multirow} % объединение ячеек таблиц по вертикали
\usepackage{afterpage} % вставка материала после текущей страницы
\usepackage[font={normal}]{caption} % настройка подписей к рисункам и таблицам
\usepackage[onehalfspacing]{setspace} % полуторный интервал
\usepackage{fancyhdr} % установка колонтитулов
\usepackage{listings} % поддержка вставки исходного кода

% Создание нового типа столбца для выравнивания содержимого по центру
\newcommand{\PreserveBackslash}[1]{\let\temp=\\#1\let\\=\temp}
\newcolumntype{C}[1]{>{\PreserveBackslash\centering}p{#1}}

% Настройка стиля страницы
\pagestyle{fancy}      % Использование стиля "fancy" для оформления страниц
\fancyhf{}              % Очистка текущих значений колонтитулов
\fancyfoot[C]{\thepage} % Установка номера страницы в нижнем колонтитуле по центру
\renewcommand{\headrulewidth}{0pt} % Удаление разделительной линии в верхнем колонтитуле

% Настройка подписей к изображениям и таблицам
\captionsetup{format=hang,labelsep=period}

% Использование полужирного начертания для векторов
\let\vec=\mathbf

% Настройка отображения теорем из amsthm amsmath
\newtheoremstyle{note}
    {3pt}
    {3pt}
    {}
    {1.25cm}
    {\bfseries}
    {.}
    { }
    {}

% ========================================================
% НАСТРОЙКА ОГЛАВЛЕНИЯ (TOC)
% ========================================================

% Установка глубины оглавления
\setcounter{tocdepth}{4}

\usepackage{tocloft}

% Вертикальные отступы между пунктами
\setlength{\cftbeforechapskip}{4pt}      % Перед главами
\setlength{\cftbeforesecskip}{2pt}       % Перед секциями
\setlength{\cftbeforesubsecskip}{1pt}    % Перед подсекциями
\setlength{\cftbeforeparaskip}{1pt}      % Перед параграфами
\setlength{\cftbeforesubparaskip}{1pt}   % Перед subsubsubsection

% Горизонтальные отступы для уровней
\renewcommand{\cftchapindent}{0em}       % Главы без отступа
\renewcommand{\cftsecindent}{1.5em}      % Секции (уровень 1)
\renewcommand{\cftsubsecindent}{3.2em}   % Подсекции (уровень 2)
\renewcommand{\cftsubsubsecindent}{5.5em} % Уровень 3 (если используется)
\renewcommand{\cftparaindent}{7.5em}     % Параграфы (уровень 4)

% Ширина полей для номеров страниц
\setlength{\cftchapnumwidth}{2em}        % Для глав
\setlength{\cftsecnumwidth}{2.5em}       % Для секций
\setlength{\cftsubsecnumwidth}{3em}      % Для подсекций
\setlength{\cftsubsubsecnumwidth}{3.5em} % Для уровня 3
\setlength{\cftparanumwidth}{4em}        % Для уровня 4

% Настройка для subsubsubsection (4-й уровень)
\makeatletter
\newcommand{\l@subsubsubsection}{\@dottedtocline{4}{7.5em}{4.5em}}
\makeatother

% ========================================================
% Новые команды
% ========================================================

% Несколько полезных блоков для выделения значимых частей текста. Пример:
% \begin{theorem}
%     Если $a$ и $b$ — положительные числа, то $a + b \geq 2\sqrt{ab}$.
%   \end{theorem}
  
%   \begin{lemma}
%     Текст леммы...
%   \end{lemma}
  
%   \begin{corollary}
%     Текст следствия...
%   \end{corollary}
  
%   \begin{note}
%     Важное замечание без номера
%   \end{note}

\theoremstyle{note}
\newtheorem{example}{Пример}
\newtheorem{claim}{Утверждение} 
\newtheorem*{note}{Замечание}
\newtheorem{theorem}{Теорема}
\newtheorem{lemma}{Лемма}
\newtheorem{corollary}{Следствие}

% Дополнительный уровень иерархии. Не попадает в оглавление
\newcommand{\subsubsubsection}[1]{%
  \paragraph{#1}%            % Жирный заголовок
  \mbox{}%                   % Пустая строка для переноса
  \par%                      % Принудительный перенос строки
  \noindent%                 % Убирает отступ у следующего абзаца
}


% Пространства
\newcommand{\R}[2]{\mathbb{R}^{#1 \times #2}}  % для матричных пространств
\newcommand{\Rn}{\mathbb{R}^n}                  % для векторных пространств
